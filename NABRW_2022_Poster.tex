%%%%%%%%%%%%%%%%%%%%%%%%%%%%%%%%%%%%%%%%%
% Jacobs Landscape Poster
% LaTeX Template
% Version 1.1 (14/06/14)
%
% Created by:
% Computational Physics and Biophysics Group, Jacobs University
% https://teamwork.jacobs-university.de:8443/confluence/display/CoPandBiG/LaTeX+Poster
%
% Further modified by:
% Nathaniel Johnston (nathaniel@njohnston.ca)
%
% This template has been downloaded from:
% http://www.LaTeXTemplates.com
%
% License:
% CC BY-NC-SA 3.0 (http://creativecommons.org/licenses/by-nc-sa/3.0/)
%
%%%%%%%%%%%%%%%%%%%%%%%%%%%%%%%%%%%%%%%%%

%----------------------------------------------------------------------------------------
%	PACKAGES AND OTHER DOCUMENT CONFIGURATIONS
%----------------------------------------------------------------------------------------

%% Path to affiliate logos
%%
%% NOTE: THIS MUST BE INCLUDED BEFORE THE MENTION OF confposter
\newcommand\logopath{C:/Users/jeffrey.neyhart/OneDrive - USDA/Documents/CranberryLab/Jeff/PostersPresentations/OtherPresentationsSlides/PresentationResources/USDA_Logos/USDALOGO-RGB.png}


%% Define the paths to the Mendeley library
\newcommand\mendlibrary{C:/Users/jeffrey.neyhart/Documents/Literature/MendeleyLibrary/library}


\documentclass[final]{beamer}

% Use the beamerposter package for laying out the poster
\usepackage[scale=1.24,orientation=portrait]{beamerposter}
% \usepackage[scale=1.24,orientation=landscape]{beamerposter}

\usetheme{confposter} % Use the confposter theme supplied with this template

\usepackage{wrapfig} % Use for wrapping figures

\usepackage[super,numbers]{natbib} % Use for fine-tunning the captions

\usepackage{tabu}


%% Configure code blocks
\usepackage{listings} % Use for code blocks
\usepackage{color}

\definecolor{dkgreen}{rgb}{0,0.6,0}
\definecolor{gray}{rgb}{0.5,0.5,0.5}
\definecolor{mauve}{rgb}{0.58,0,0.82}

\lstset{frame=tb,
  language=Java,
  aboveskip=3mm,
  belowskip=3mm,
  showstringspaces=false,
  columns=flexible,
  basicstyle={\small\ttfamily},
  numbers=none,
  numberstyle=\tiny\color{gray},
  keywordstyle=\color{blue},
  commentstyle=\color{dkgreen},
  stringstyle=\color{mauve},
  breaklines=true,
  breakatwhitespace=true,
  tabsize=3
}

% \usepackage{biblatex} % Bibliography

\setbeamercolor{block title}{fg=greenset1,bg=white} % Colors of the block titles
\setbeamercolor{block body}{fg=black,bg=white} % Colors of the body of blocks
\setbeamercolor{block alerted title}{fg=white,bg=blueset1!70} % Colors of the highlighted block titles
\setbeamercolor{block alerted body}{fg=black,bg=blueset1!10} % Colors of the body of highlighted blocks

% Many more colors are available for use in beamerthemeconfposter.sty

%-----------------------------------------------------------

% Define the column widths and overall poster size
% To set effective sepwid, onecolwid and twocolwid values, first choose how many columns you want and how much separation you want between columns
% In this template, the separation width chosen is 0.02 of the paper width and a 3-column layout
% onecolwid should therefore be (1-(ncol + 1)*sepwid)/ncol e.g. (1-(3+1)*0.024)/3 = 0.3013333
% Set twocolwid to be (2*onecolwid)+sepwid = 0.6266666
% Set threecolwid to be (3*onecolwid)+2*sepwid = 0.9279999

\newlength{\sepwid}
\newlength{\onecolwid}
\newlength{\twocolwid}
\newlength{\threecolwid}

%% Poster size
\setlength{\paperwidth}{33in} % A0 width: 46.8in
\setlength{\paperheight}{46in} % A0 height: 33.1in


\setlength{\sepwid}{0.024\paperwidth} % Separation width (white space) between columns
\setlength{\onecolwid}{0.3013333\paperwidth} % Width of one column
\setlength{\twocolwid}{0.6266666\paperwidth} % Width of two columns
\setlength{\threecolwid}{0.9279999\paperwidth} % Width of three columns
\setlength{\topmargin}{-1in} % Reduce the top margin size
%-----------------------------------------------------------

\usepackage{graphicx}  % Required for including images

\usepackage{booktabs} % Top and bottom rules for tables

\usepackage{array}

%----------------------------------------------------------------------------------------
%	TITLE SECTION
%----------------------------------------------------------------------------------------

\title{Location, location, location: Identifying precise, repeatable, and representative real estate for barley uniform nurseries} % Poster title
% Add \\ to induce new line

\vspace{1cm}

% Author(s)
\author{ Jeffrey L. Neyhart\textsuperscript{1,2*}, Lucia Gutierrez\textsuperscript{3}, and Kevin P. Smith\textsuperscript{2} }

% Affiliations
\institute{
  \textsuperscript{1}USDA-ARS, Plant Science Research Unit, St. Paul, MN.;
  \textsuperscript{2}Dept. of Agronomy and Plant Genetics, Univ. of Minnesota, St. Paul, MN.;
  \textsuperscript{3}Dept. of Agronomy, Univ. of Wisconsin-Madison, Madison, WI.
}

% *Contact: Email - \href{mailto:jeffrey.neyhart@usda.gov}{jeffrey.neyhart@usda.gov}


%----------------------------------------------------------------------------------------

\begin{document}

\addtobeamertemplate{block end}{}{\vspace*{2ex}} % White space under blocks
\addtobeamertemplate{block alerted end}{}{\vspace*{2ex}} % White space under highlighted (alert) blocks

\setlength{\belowcaptionskip}{2ex} % White space under figures
\setlength\belowdisplayshortskip{2ex} % White space under equations

\begin{frame}[t] % The whole poster is enclosed in one beamer frame

\begin{columns}[t] % The whole poster consists of three major columns, the second of which is split into two columns twice
% the [t] option aligns each column's content to the top

%----------------------------------------------------------------------------------------

\begin{column}{\sepwid}\end{column} % Empty spacer column

\begin{column}{\onecolwid} % The first column


%----------------------------------------------------------------------------------------
%	TAKEAWAYS
%----------------------------------------------------------------------------------------

\setbeamercolor{block alerted title}{fg=white,bg=orangeset1!70} % Change the alert block title colors
\setbeamercolor{block alerted body}{fg=black,bg=orangeset1!10} % Change the alert block body colors

\begin{alertblock}{\Large{Takeaways}}

% \begin{large}
\begin{itemize}
  \item \textbf{Not all locations in uniform nurseries are ideal for phenotyping all traits}
  \vspace{0.5cm}
  \item \textbf{We developed an optimization method to select the best locations for phenotyping}
  \vspace{0.5cm}
  \item \textbf{Compared to all locations, optimization maintained or increased phenotype data quality while reducing costs}
\end{itemize}
% \end{large}


\end{alertblock}

%----------------------------------------------------------------------------------------
%	INTRODUCTION
%----------------------------------------------------------------------------------------

\begin{block}{Introduction}

Breeding programs test genotypes performance in costly multi-location trials

\vspace{0.5cm}

Test locations can be evaluated using three criteria\cite{Yan2011}:
\begin{itemize}
  \item \textbf{Precision}: the power to discriminate genotypes
  \item \textbf{Repeatability}: the consistency of genotype performance
  \item \textbf{Representativeness}: the similarity of a location to a mega-environment
\end{itemize}

\vspace{0.5cm}

Can we improve the efficiency of regional nurseries by determining the best locations for phenotyping?

\vspace{0.5cm}

\textbf{Objectives:}
\begin{itemize}
  \item Compare test location precision, repeatability, and representativeness in two regional barley nurseries
  \item Determine optimized test locations for phenotyping multiple agronomic and malting quality traits
\end{itemize}


\end{block}



% %----------------------------------------------------------------------------------------
% %	OBJECTIVES OR QUESTIONS
% %----------------------------------------------------------------------------------------
%
% % Set the color
% \setbeamercolor{block alerted title}{fg=white,bg=blueset1!70} % Change the alert block title colors
% \setbeamercolor{block alerted body}{fg=black,bg=blueset1!10} % Change the alert block body colors
%
% \begin{alertblock}{\large{Objective}}
%
% \begin{itemize}
%   \item{\textbf{Identify genomic regions associated with local environmental factors in wild cranberry populations.}}
%   % \item{\textbf{Determine candidate abiotic stress tolerance loci as potential selection targets.}}
%   % \item \textbf{What genomic regions are associated with local environmental conditions in wild cranberry?}
%   % \item What is the frequency and occurrance of putatively adaptive loci in native cranberry selections?
% \end{itemize}
%
%
% \end{alertblock}


% \vspace{1cm}


%----------------------------------------------------------------------------------------
%	MATERIALS AND MATERIALS
%----------------------------------------------------------------------------------------

\begin{block}{Materials and Methods}

We used data from the Mississippi Valley (MVN) and Western Regional (WRN) uniform nurseries

% Figure with sampling locations
\begin{figure}
  \begin{center}
    \includegraphics[width=0.90\linewidth]{figures/nursery_location_map.png}
    % \caption{Locations in the MVN, WRN, or both nurseries.}
  \end{center}
\end{figure}

\begin{footnotesize}

%% Table of parameters
\begin{table}[H]
\centering
\begin{tabular}{>{\raggedright\arraybackslash}p{5in}>{\centering\arraybackslash}p{2in}>{\centering\arraybackslash}p{2in}}
\toprule
% \textbf{Parameter} & \textbf{Mississippi Valley Nursery} & \textbf{Western Regional Nursery}\\
\textbf{Parameter} & \textbf{MVN} & \textbf{WRN}\\
\midrule
Locations & 20 & 32\\
Years & 25 & 23\\
Environments (Loc.-Yr.) & 175 & 251\\
Genotypes & 401 & 393\\
Traits & 19 & 18\\
\bottomrule
\end{tabular}
\end{table}

\end{footnotesize}



\vspace{1cm}

Three statistics were estimated per location:
\begin{itemize}
  \item \textbf{Precision} - variance of a genotype mean ($V_{\bar{Y}}$)\cite{Bernardo2010} (lower $V_{\bar{Y}}$ means more precise)
  \item \textbf{Repeatability} - genetic correlation ($\rho_G$) across years (higher $\rho_G$ means more repeatable)
  \item \textbf{Representativeness} - $\rho_G$ between a location and a mega-environment (higher $\rho_G$ means more representative)
\end{itemize}

We defined an optimization algorithm to select locations that maximized these statistics for multiple traits



% End the M&M block
\end{block}



%----------------------------------------------------------------------------------------

\end{column} % End of the first column (i.e. Introduction, objectives, and methods)

%----------------------------------------------------------------------------------------

% Begin the results columns

\begin{column}{\sepwid}\end{column} % Empty spacer column

\begin{column}{\twocolwid} % Begin a column which is two columns wide (column 2)


% Removing the two-columns within the middle column structure
% \begin{columns}[t,totalwidth=\twocolwid] % Split up the two columns wide column
%
% \begin{column}{\onecolwid}\vspace{-.6in} % The first column within column 2 (column 2.1)

%----------------------------------------------------------------------------------------
%	RESULTS
%----------------------------------------------------------------------------------------

\begin{block}{Results}


%----------------------------------------------------------------------------------------

% \vspace{0.5cm}
%
% %% Single column for now
% % One column
% \begin{columns}[t,totalwidth=\twocolwid] % Split up the two columns wide column
%
% % A single column
% \begin{column}{\twocolwid}
%
%
% % Figure 2
%
% \begin{center}
%   \begin{figure}
%     \includegraphics[width=1\linewidth]{figures/figure2.png}
%     \caption{A figure!}
%   \end{figure}
% \end{center}
%
%
% % End the column
% \end{column}
%
% % End of column split
% \end{columns}
%




%----------------------------------------------------------------------------------------

\vspace{0.5cm}

%% Columns for figure 3


%% Split into two columns
\begin{columns}[t,totalwidth=\twocolwid] % Split up the two columns wide column


% Second column
\begin{column}{0.75\twocolwid}

% Figure 3
\begin{center}
  \begin{figure}
    \includegraphics[width=0.95\linewidth]{figures/figure3.jpg}
    % \caption{A figure!}
  \end{figure}
\end{center}

\end{column}


% First column
\begin{column}{0.23\twocolwid}

The best test locations are denoted by \textbf{larger points in the upper-right corner} of each plot
% \begin{itemize}
%   \item Precision (\textit{x}-axis; increases left-to-right)
%   \item Repeatability (\textit{y}-axis; increases bottom-to-top)
%   \item Representativeness (point size; increases with larger points)
% \end{itemize}

\vspace{1cm}


Some locations are ideal for multiple traits: BTT (Bottineau, ND); PLL (Pullman, WA)

\vspace{1cm}

For other locations, only some traits are reliably phenotyped: MRR (Morris, MN); ABR (Aberdeen, ID)

\vspace{1cm}

\textbf{Most test locations are not ideal for phenotyping all traits}

\end{column}


% First column
\begin{column}{0.02\twocolwid}
\end{column}

\end{columns}

\vspace{4cm}

%----------------------------------------------------------------------------------------



%% Columns for figure 4

%% Split into two columns
\begin{columns}[t,totalwidth=\twocolwid] % Split up the two columns wide column

% First column
\begin{column}{0.48\twocolwid}


% Figure 4
\begin{center}
  \begin{figure}
    \includegraphics[width=0.75\linewidth]{figures/figure4a.jpg}
    % \caption{A figure!}
  \end{figure}
\end{center}

\textbf{Fewer, optimized locations led to similar or better phenotype data quality compared to all or random sets of locations}

\vspace{0.5cm}




\end{column}

% Spacer
\begin{column}{0.02\twocolwid}
\end{column}


% Second column
\begin{column}{0.48\twocolwid}

% Figure 5
\begin{center}
  \begin{figure}
    \includegraphics[width=0.85\linewidth]{figures/figure5.jpg}
    % \caption{A figure!}
  \end{figure}
\end{center}


\vspace{1cm}


\begingroup\fontsize{22}{24}\selectfont

\begin{tabu} to \linewidth {>{\raggedright}X>{\centering}X>{\centering}X>{\centering}X>{\centering}X>{\centering}X}
\toprule
\multicolumn{1}{c}{ } & \multicolumn{1}{c}{ } & \multicolumn{1}{c}{ } & \multicolumn{3}{c}{\textbf{\% Improvement Vs. All Locations}} \\
\cmidrule(l{3pt}r{3pt}){4-6}
\textbf{Nursery} & \textbf{Selected Agro. Trait Loc.} & \textbf{Selected Malt Qual. Trait Loc.} & \textbf{Precision} & \textbf{Repeat-ability} & \textbf{Represent-ativeness}\\
\midrule
MVN & 5 & 2 & -7.48 & 33.3 & 15.0\\
WRN & 5 & 3 & 2.98 & 16.0 & 8.05\\
\bottomrule
\end{tabu}
\endgroup{}


\vspace{1cm}

\textbf{Optimization generally increased phenotype data quality while reducing the number of locations by 50-75\%}


\vspace{1cm}


% %% Figure of the second module
% \begin{alertblock}{Implications}
%
% \begin{itemize}
%   \item{Text}
%   \item{Text}
%   \item{Text}
% \end{itemize}
%
% \end{alertblock}

\end{column}

% Spacer
\begin{column}{0.02\twocolwid}
\end{column}



\end{columns}



%-------------------------------

% % Insert an alert block that spans the whole width
%
% % Set the color
% \setbeamercolor{block alerted title}{fg=white,bg=orangeset1!70} % Change the alert block title colors
% \setbeamercolor{block alerted body}{fg=black,bg=orangeset1!10} % Change the alert block body colors
%
% \begin{alertblock}{}
%
% % Divide into columns
% % Now split into two columns
% \begin{columns}[t,totalwidth=\twocolwid] % Split up the two columns wide column
%
% \begin{column}{\onecolwid} %
%
% %%% Fill in here
%
% \end{column} % End of column
%
% %%-----------------------------------
%
% \begin{column}{\onecolwid} % The second column within column 2 (column 2.2)
%
% %%% Fill in here
%
%
% \end{column} % End of column
%
% \end{columns} % End of the split of column
%
% \end{alertblock}

%--------------------------------


% End the results block
\end{block}

%----------------------------


%% Add some space
\vspace{2cm}


%----------------------------------------------------------------------------------------
% Split into columns
\begin{columns}[t,totalwidth=\twocolwid] % Split up the two columns wide column
%----------------------------------------------------------------------------------------

%----------------------------------------------------------------------------------------
%	DISCUSSION/CONCLUSIONS
%----------------------------------------------------------------------------------------

\begin{column}{0.48\twocolwid}

% Set the color
\setbeamercolor{block alerted title}{fg=white,bg=orangeset1!70} % Change the alert block title colors
\setbeamercolor{block alerted body}{fg=black,bg=orangeset1!10} % Change the alert block body colors

\begin{alertblock}{Read more / reach out}

\begin{minipage}{0.3\linewidth}
\begin{figure}[H]
  \includegraphics[width=1\linewidth]{figures/cs_qr_code.png}
\end{figure}
\end{minipage} \hfill
\begin{minipage}{0.7\linewidth}
Read our paper in \textit{Crop Science}
\end{minipage}


\textbf{Contact:}
\begin{itemize}
  \item Email: jeffrey.neyhart@usda.gov
  \item LinkedIn: linkedin.com/in/jeffneyhart
  \item GitHub: github.com/neyhartj
\end{itemize}



% % Split into columns
% \begin{columns}[t,totalwidth=\onecolwid] % Split up the two columns wide column
%
% \begin{column}{0.5\onecolwid}
%
%
%
% % QR code
% \begin{center}
%
% Read our paper in \textit{Crop Science}:
%
% \begin{figure}
%   \includegraphics[width=0.5\linewidth]{figures/cs_qr_code.png}
% \end{figure}
%
% \end{center}
%
%
% \end{column}
%
% \begin{column}{0.5\onecolwid}
%
%
% % QR code
% \begin{center}
%
% Access the code for this analysis:
%
% \begin{figure}
%   \includegraphics[width=0.5\linewidth]{figures/github_qr_code.png}
% \end{figure}
%
% \end{center}
%
%
% \end{column}
%
% \end{columns}


\end{alertblock}


\end{column}


% Spacer
\begin{column}{0.02\twocolwid}
\end{column}



%----------------------------------------------------------------------------------------

%% Open an Acknowledgements/Reference column

\begin{column}{0.48\twocolwid}



%----------------------------------------------------------------------------------------
%	ACKNOWLEDGEMENTS
%----------------------------------------------------------------------------------------


% Change the font of the block title for the next two sections
% \setbeamerfont{block title}{size=\small}

\begin{block}{\large{Acknowledgements}}

\begin{footnotesize}

We thank C. Evans, C. Martens, and S. Yang (USDA-ARS) for providing data from the nursery reports. This research was supported by USDA National Institute of Food and Agriculture Grant 2018-67013-27620 and USDA-ARS project 8042-21000-279-00D. USDA is an equal opportunity provider and employer.

\end{footnotesize}

\end{block}


%----------------------------------------------------------------------------------------
%	REFERENCES
%----------------------------------------------------------------------------------------

\begin{block}{\large{References}}

% \nocite{*} % Insert publications even if they are not cited in the poster
% ^ removed this so list only those publications that are cited

% \setbeamercolor{

\begin{footnotesize}

\bibliographystyle{supporting_files/posterbibstyle}
\bibliography{\mendlibrary}


\end{footnotesize}

% End the reference block
\end{block}


% %----------------------------------------------------------------------------------------
% %	CONTACT INFORMATION
% %----------------------------------------------------------------------------------------
%
% \setbeamercolor{block alerted title}{fg=white,bg=orangeset1!70} % Change the alert block title colors
% \setbeamercolor{block alerted body}{fg=black,bg=orangeset1!10} % Change the alert block body colors
%
% \begin{alertblock}{\large{\textsuperscript{*}Contact Information}}
%
% \begin{itemize}
% \item Email: \href{mailto:neyha001@umn.edu}{neyha001@umn.edu}
% \item Twitter: \href{https://twitter.com/neyhartje}{@neyhartje}
% \item Web: \href{http://smithlab.cfans.umn.edu/}{smithlab.cfans.umn.edu}
% \end{itemize}
%
% \end{alertblock}


% End the acknowlegements/references column
\end{column}

% Spacer
\begin{column}{0.02\twocolwid}
\end{column}


% End the two bottom columns
\end{columns}
%----------------------------------------------------------------------------------------

%----------------------------------------------------------------------------------------
% End the second main column
\end{column} % End of the third column
%----------------------------------------------------------------------------------------



\end{columns} % End of all the columns in the poster

\end{frame} % End of the enclosing frame

\end{document} % End the poster document
